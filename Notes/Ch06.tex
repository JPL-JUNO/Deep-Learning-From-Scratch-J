\chapter{与学习相关的技巧}
本章将介绍神经网络的学习中的一些重要观点,主题涉及寻找最优权重
参数的最优化方法、权重参数的初始值、超参数的设定方法等。此外,为了
应对过拟合,本章还将介绍权值衰减、Dropout等正则化方法,并进行实现。
\section{参数的更新}
神经网络的学习的目的是找到使损失函数的值尽可能小的参数。这是寻
找最优参数的问题,解决这个问题的过程称为\textbf{最优化}(optimization)。遗憾的是,
神经网络的最优化问题非常难。这是因为参数空间非常复杂,无法轻易找到
最优解(无法使用那种通过解数学式一下子就求得最小值的方法)。而且,在
深度神经网络中,参数的数量非常庞大,导致最优化问题更加复杂。

使用参数的梯度,沿梯度方向更新参数,并重复这个步骤多次,从而逐渐靠
近最优参数,这个过程称为\textbf{随机梯度下降法}(stochastic gradient descent),
简称\textbf{SGD}。
\subsection{SGD}
用数学式可以将SGD写成如下式:
\begin{equation*}
    \bm{W}\leftarrow \bm{W}-\eta\frac{\partial L}{\partial \bm{W}}
\end{equation*}
SGD是朝着梯度方向只前进一定距离的简单方法。

\subsection{SGD的缺点}
虽然SGD简单,并且容易实现,但是在解决某些问题时可能没有效率。
\begin{equation*}
    z = \frac{1}{20}x^2+y^2
\end{equation*}
上式表示的函数是向 $x$ 轴方向延伸的“碗”状函数。

SGD的缺点是,如果函数的形状非均向(anisotropic),比如呈延伸状,搜索
的路径就会非常低效。因此,我们需要比单纯朝梯度方向前进的SGD更聪
明的方法。\important{SGD低效的根本原因是,梯度的方向并没有指向最小值的方向}。

\subsection{Momentum}
\begin{subequations}
    \begin{align}
        \bm{v} & \leftarrow  \alpha \bm{v}-\eta\frac{\partial L}{\partial \bm{W}} \label{eq11-a} \\
        \bm{W} & \leftarrow \bm{W}+\bm{v}\label{eq11-b}
    \end{align}
\end{subequations}
这里新出现了一个变量$\bm{v}$,对应物理上的速度。
\autoref{eq11-a}表示了物体在梯度方向上受力,在这个力的作用下,物体的速度增
加这一物理法则。如\autoref{Momentum-The ball rolls on an incline}所示,Momentum方法给人的感觉就像是小球在
地面上滚动。

式\autoref{eq11-a}中有$\alpha \bm{v}$这一项。在物体不受任何力时,该项承担使物体逐渐减
速的任务($\alpha$设定为0.9之类的值),对应物理上的地面摩擦或空气阻力。
\figures{Momentum-The ball rolls on an incline}
\subsection{AdaGrad}
在关于学习率的有效技巧中,有一种被称为\textbf{学习率衰减}(learning rate
decay)的方法,即随着学习的进行,使学习率逐渐减小。实际上,一开始“多”
学,然后逐渐“少”学的方法,在神经网络的学习中经常被使用。

逐渐减小学习率的想法,相当于将“全体”参数的学习率值一起降低。
而AdaGrad进一步发展了这个想法,针对“一个一个”的参数,赋予其“定
制”的值。AdaGrad会为参数的每个元素适当地调整学习率,与此同时进行学习。
\begin{subequations}
    \begin{align}
        \bm{h} & \leftarrow  \bm{h} + \frac{\partial L}{\partial \bm{W}} \odot \frac{\partial L}{\partial \bm{W}} \label{eq12-a} \\
        \bm{W} & \leftarrow \bm{W} -\eta \frac{1}{\sqrt{\bm{h}}}\frac{\partial L}{\partial \bm{W}} \label{eq12-b}
    \end{align}
\end{subequations}
式中,$\odot$表示对应矩阵元素的乘法,在更新参数时,通过乘以$\frac{1}{\sqrt{\bm{h}}}$
,就可以调整学习的尺度。这意味着,
参数的元素中变动较大(被大幅更新)的元素的学习率将变小。也就是说,
\important{可以按参数的元素进行学习率衰减,使变动大的参数的学习率逐渐减小}。
\begin{tcolorbox}
    \important{AdaGrad会记录过去所有梯度的平方和}。因此,学习越深入,更新
    的幅度就越小。实际上,如果无止境地学习,更新量就会变为 0,
    完全不再更新。为了改善这个问题,可以使用 RMSProp方法。
    RMSProp方法并不是将过去所有的梯度一视同仁地相加,而是逐渐
    地遗忘过去的梯度,在做加法运算时将新梯度的信息更多地反映出来。
    这种操作从专业上讲,称为“指数移动平均”,呈指数函数式地减小
    过去的梯度的尺度。
\end{tcolorbox}

\subsection{Adam}
\href{https://arxiv.org/abs/1412.6980v8}{Adam}是2015年提出的新方法。它的理论有些复杂,直观地讲,就是融
合了Momentum和AdaGrad的方法。通过组合前面两个方法的优点,有望
实现参数空间的高效搜索。此外,进行超参数的“偏置校正”也是Adam的特征。

\begin{tcolorbox}
    Adam会设置3个超参数。一个是学习率(论文中以$\alpha$出现),另外两
    个是一次momentum系数$\beta_1$和二次momentum系数$\beta_2$。根据论文,
    标准的设定值是$\beta_1$为0.9,$\beta_2$ 为0.999。设置了这些值后,大多数情
    况下都能顺利运行。
\end{tcolorbox}
\subsection{使用哪种更新方法呢}
\figures{Comparison of Optimization Methods}

如\autoref{Comparison of Optimization Methods}所示,根据使用的方法不同,参数更新的路径也不同。只看这
个图的话,AdaGrad似乎是最好的,不过也要注意,结果会根据要解决的问
题而变。并且,很显然,超参数(学习率等)的设定值不同,结果也会发生变化。

非常遗憾,(目前)并不存在能在所有问题中都表现良好
的方法。这4种方法各有各的特点,都有各自擅长解决的问题和不擅长解决
的问题。

\subsection{基于MNIST数据集的更新方法的比较}
\section{权重的初始值}
\section{Batch Normalization}
\section{正则化}
\section{超参数的验证}