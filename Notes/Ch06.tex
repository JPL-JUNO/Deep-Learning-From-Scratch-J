\chapter{与学习相关的技巧}
本章将介绍神经网络的学习中的一些重要观点,主题涉及寻找最优权重
参数的最优化方法、权重参数的初始值、超参数的设定方法等。此外,为了
应对过拟合,本章还将介绍权值衰减、Dropout等正则化方法,并进行实现。
\section{参数的更新}
神经网络的学习的目的是找到使损失函数的值尽可能小的参数。这是寻
找最优参数的问题,解决这个问题的过程称为\textbf{最优化}(optimization)。遗憾的是,
神经网络的最优化问题非常难。这是因为参数空间非常复杂,无法轻易找到
最优解(无法使用那种通过解数学式一下子就求得最小值的方法)。而且,在
深度神经网络中,参数的数量非常庞大,导致最优化问题更加复杂。

使用参数的梯度,沿梯度方向更新参数,并重复这个步骤多次,从而逐渐靠
近最优参数,这个过程称为\textbf{随机梯度下降法}(stochastic gradient descent),
简称\textbf{SGD}。
\subsection{SGD}
用数学式可以将SGD写成如下式:
\begin{equation*}
    \bm{W}\leftarrow \bm{W}-\eta\frac{\partial L}{\partial \bm{W}}
\end{equation*}
SGD是朝着梯度方向只前进一定距离的简单方法。

\subsection{SGD的缺点}
虽然SGD简单,并且容易实现,但是在解决某些问题时可能没有效率。
\begin{equation*}
    z = \frac{1}{20}x^2+y^2
\end{equation*}
上式表示的函数是向 $x$ 轴方向延伸的“碗”状函数。

SGD的缺点是,如果函数的形状非均向(anisotropic),比如呈延伸状,搜索
的路径就会非常低效。因此,我们需要比单纯朝梯度方向前进的SGD更聪
明的方法。\important{SGD低效的根本原因是,梯度的方向并没有指向最小值的方向}。

\subsection{Momentum}
\begin{subequations}
    \begin{align}
        \bm{v} & \leftarrow  \alpha \bm{v}-\eta\frac{\partial L}{\partial \bm{W}} \label{eq11-a} \\
        \bm{W} & \leftarrow \bm{W}+\bm{v}\label{eq11-b}
    \end{align}
\end{subequations}
这里新出现了一个变量$\bm{v}$,对应物理上的速度。
\autoref{eq11-a}表示了物体在梯度方向上受力,在这个力的作用下,物体的速度增
加这一物理法则。如\autoref{Momentum-The ball rolls on an incline}所示,Momentum方法给人的感觉就像是小球在
地面上滚动。

式\autoref{eq11-a}中有$\alpha \bm{v}$这一项。在物体不受任何力时,该项承担使物体逐渐减
速的任务($\alpha$设定为0.9之类的值),对应物理上的地面摩擦或空气阻力。
\figures{Momentum-The ball rolls on an incline}
\subsection{AdaGrad}
在关于学习率的有效技巧中,有一种被称为\textbf{学习率衰减}(learning rate
decay)的方法,即随着学习的进行,使学习率逐渐减小。实际上,一开始“多”
学,然后逐渐“少”学的方法,在神经网络的学习中经常被使用。

逐渐减小学习率的想法,相当于将“全体”参数的学习率值一起降低。
而AdaGrad进一步发展了这个想法,针对“一个一个”的参数,赋予其“定
制”的值。AdaGrad会为参数的每个元素适当地调整学习率,与此同时进行学习。
\begin{subequations}
    \begin{align}
        \bm{h} & \leftarrow  \bm{h} + \frac{\partial L}{\partial \bm{W}} \odot \frac{\partial L}{\partial \bm{W}} \label{eq12-a} \\
        \bm{W} & \leftarrow \bm{W} -\eta \frac{1}{\sqrt{\bm{h}}}\frac{\partial L}{\partial \bm{W}} \label{eq12-b}
    \end{align}
\end{subequations}
式中,$\odot$表示对应矩阵元素的乘法,在更新参数时,通过乘以$\frac{1}{\sqrt{\bm{h}}}$
,就可以调整学习的尺度。这意味着,
参数的元素中变动较大(被大幅更新)的元素的学习率将变小。也就是说,
\important{可以按参数的元素进行学习率衰减,使变动大的参数的学习率逐渐减小}。
\begin{tcolorbox}
    \important{AdaGrad会记录过去所有梯度的平方和}。因此,学习越深入,更新
    的幅度就越小。实际上,如果无止境地学习,更新量就会变为 0,
    完全不再更新。为了改善这个问题,可以使用 RMSProp方法。
    RMSProp方法并不是将过去所有的梯度一视同仁地相加,而是逐渐
    地遗忘过去的梯度,在做加法运算时将新梯度的信息更多地反映出来。
    这种操作从专业上讲,称为“指数移动平均”,呈指数函数式地减小
    过去的梯度的尺度。
\end{tcolorbox}

\subsection{Adam}
\href{https://arxiv.org/abs/1412.6980v8}{Adam}是2015年提出的新方法。它的理论有些复杂,直观地讲,就是融
合了Momentum和AdaGrad的方法。通过组合前面两个方法的优点,有望
实现参数空间的高效搜索。此外,进行超参数的“偏置校正”也是Adam的特征。

\begin{tcolorbox}
    Adam会设置3个超参数。一个是学习率(论文中以$\alpha$出现),另外两
    个是一次momentum系数$\beta_1$和二次momentum系数$\beta_2$。根据论文,
    标准的设定值是$\beta_1$为0.9,$\beta_2$ 为0.999。设置了这些值后,大多数情
    况下都能顺利运行。
\end{tcolorbox}
\subsection{使用哪种更新方法呢}
\figures{Comparison of Optimization Methods}

如\autoref{Comparison of Optimization Methods}所示,根据使用的方法不同,参数更新的路径也不同。只看这
个图的话,AdaGrad似乎是最好的,不过也要注意,结果会根据要解决的问
题而变。并且,很显然,超参数(学习率等)的设定值不同,结果也会发生变化。

非常遗憾,(目前)并不存在能在所有问题中都表现良好
的方法。这4种方法各有各的特点,都有各自擅长解决的问题和不擅长解决
的问题。

\subsection{基于MNIST数据集的更新方法的比较}
\section{权重的初始值}
在神经网络的学习中,权重的初始值特别重要。实际上,设定什么样的
权重初始值,经常关系到神经网络的学习能否成功。

\subsection{可以将权重初始值设为0吗}
权值衰减(weights decay)就是一种以减小权重参数的值为目的进行学习
的方法。通过减小权重参数的值来抑制过拟合的发生。从结
论来说,将权重初始值设为0不是一个好主意。事实上,将权重初始值设为
0的话,将无法正确进行学习。

为了防止“权重均一化”
(严格地讲,是为了瓦解权重的对称结构),必须随机生成初始值。
\subsection{隐藏层的激活值的分布}
观察隐藏层的激活值\footnote{这里我们将激活函数的输出数据称为“激活值”,但是有的文献中会将在层之间流动的数据也称为“激
    活值”。}(激活函数的输出数据)的分布,可以获得很多启
发。
\figures{The distribution of the activation value of each layer when using a Gaussian distribution with a standard deviation of 1 as the initial weight value}
从\autoref{The distribution of the activation value of each layer when using a Gaussian distribution with a standard deviation of 1 as the initial weight value}可知,各层的激活值呈偏向0和1的分布。这里使用的sigmoid
函数是S型函数,随着输出不断地靠近0(或者靠近1),它的导数的值逐渐接
近0。因此,偏向0和1的数据分布会造成反向传播中梯度的值不断变小,最
后消失。这个问题称为梯度消失(gradient vanishing)。层次加深的深度学习
中,梯度消失的问题可能会更加严重。

\begin{tcolorbox}
    各层的激活值的分布都要求有适当的广度。为什么呢?因为通过
    在各层间传递多样性的数据,神经网络可以进行高效的学习。反
    过来,如果传递的是有所偏向的数据,就会出现梯度消失或者“表
    现力受限”的问题,导致学习可能无法顺利进行。
\end{tcolorbox}

Xavier的论文中,为了使各层的激活值呈现出具有相同广度的分布,推导了合适的权重尺度。推导出的结论是,如果前一层的节点数为$n$,则初始
值使用标准差为$\sqrt{\frac{1}{n}}$
的分布。

\figures{Xavier initial value}

使用Xavier初始值后的结果如\autoref{Distribution of activation values of each layer when using Xavier initial value as weight initial value}所示。从这个结果可知,越是后
面的层,图像变得越歪斜,但是呈现了比之前更有广度的分布。因为各层间
传递的数据有适当的广度,所以sigmoid函数的表现力不受限制,有望进行
高效的学习。

\begin{tcolorbox}
    \autoref{Distribution of activation values of each layer when using Xavier initial value as weight initial value}的分布中,后面的层的分布呈稍微歪斜的形状。如果用tanh
    函数(双曲线函数)代替 sigmoid函数,这个稍微歪斜的问题就能得
    到改善。实际上,使用 tanh函数后,会呈漂亮的吊钟型分布。tanh
    函数和sigmoid函数同是S型曲线函数,但tanh函数是关于原点$(0, 0)$
    对称的S型曲线,而 sigmoid函数是关于$(x, y)=(0, 0.5)$对称的S型曲
    线。\important{众所周知,用作激活函数的函数最好具有关于原点对称的性质}。
\end{tcolorbox}

\figures{Distribution of activation values of each layer when using Xavier initial value as weight initial value}

\subsection{ReLU的权重初始值}
Xavier 初始值是以激活函数是线性函数为前提而推导出来的。因为
sigmoid函数和 tanh函数左右对称,且中央附近可以视作线性函数,所以适
合使用Xavier初始值。但当激活函数使用ReLU时,一般推荐使用ReLU专
用的初始值,也就是Kaiming He等人推荐的初始值,也称为“He初始值”。
当前一层的节点数为$n$时,He 初始值使用标准差为$\sqrt{\frac{2}{n}}$
的高斯分布。

总结一下,当激活函数使用ReLU时,权重初始值使用He初始值,当
激活函数为 sigmoid或 tanh等S型曲线函数时,初始值使用Xavier初始值。
这是目前的最佳实践。
\subsection{基于MNIST数据集的权重初始值的比较}
\figures{Comparison of weight initial values based on MNIST dataset}
这个实验中,神经网络有5层,每层有100个神经元,激活函数使用的
是ReLU。从\autoref{Comparison of weight initial values based on MNIST dataset}的结果可知,std = 0.01时完全无法进行学习。这和刚
才观察到的激活值的分布一样,是因为正向传播中传递的值很小(集中在0
附近的数据)。因此,逆向传播时求到的梯度也很小,权重几乎不进行更新。
相反,当权重初始值为Xavier初始值和He初始值时,学习进行得很顺利。
并且,我们发现He初始值时的学习进度更快一些。
\section{Batch Normalization}
在上一节,我们观察了各层的激活值分布,并从中了解到如果设定了合
适的权重初始值,则各层的激活值分布会有适当的广度,从而可以顺利地进
行学习。那么,为了使各层拥有适当的广度,“强制性”地调整激活值的分布
会怎样呢?实际上,Batch Normalization 方法就是基于这个想法而产生的。
\subsection{Batch Normalization 的算法}
什么Batch Norm这么惹人注目呢?因为Batch Norm有以下优点。
\begin{itemize}
    \item 可以使学习快速进行(可以增大学习率)。
    \item 不那么依赖初始值(对于初始值不用那么神经质)。
    \item 抑制过拟合(降低Dropout等的必要性)。
\end{itemize}
考虑到深度学习要花费很多时间,第一个优点令人非常开心。另外,后
两点也可以帮我们消除深度学习的学习中的很多烦恼。

Batch Norm 的\important{思路是调整各层的激活值分布使其拥有适当的广度}。为此,要向神经网络中插入对数据分布进行正规化的层,即Batch
Normalization层(下文简称Batch Norm层),如\autoref{An example of a neural network using Batch Normalization}所示。
\figures{An example of a neural network using Batch Normalization}
Batch Norm,顾名思义,以进行学习时的mini-batch为单位,按mini-batch进行正规化。具体而言,就是进行使数据分布的均值为0、方差为1的
正规化。用数学式表示的话,如下所示:
\begin{equation*}
    \begin{aligned}
        \mu_B      & \leftarrow \frac{1}{m}\sum\limits_{i=1}^mx_i             \\
        \sigma_B^2 & \leftarrow \frac{1}{m-1}\sum\limits_{i=1}^m(x_i-\mu_B)^2 \\
        \hat{x}_i  & \leftarrow \frac{x_i-\mu_B}{\sqrt{\sigma_B^2+\epsilon}}  \\
    \end{aligned}
\end{equation*}

Batch Norm层会对正规化后的数据进行缩放和平移的变换,用
数学式可以如下表示:
\begin{equation*}
    y_i\leftarrow \gamma\hat{x}_i+\beta
\end{equation*}
这里,$\gamma$和$\beta$是参数。一开始$\gamma=1$,$\beta=0$,然后再通过学习调整到合
适的值。

\figures{Computational graph of Batch Normalization}
如果使用\autoref{Computational graph of Batch Normalization}的计算图来思考的话,Batch Norm的反向传播或许也能比较
轻松地推导出来。Frederik Kratzert 的博客“\href{https://kratzert.github.io/2016/02/12/understanding-the-gradient-flow-through-the-batch-normalization-layer.html}{Understanding the backward pass through Batch Normalization Layer}”里有详细说明。

\subsection{Batch Normalization的评估}
\section{正则化}
\section{超参数的验证}