\chapter{卷积神经网络}
本章的主题是卷积神经网络(Convolutional Neural Network,CNN)。
CNN被用于图像识别、语音识别等各种场合。
\section{整体结构}
CNN和之前介绍的神经网络一样,可以像乐高积木一样通过组装层来构建。不过,
CNN中新出现了卷积层(Convolution层)和池化层(Pooling层)。

之前介绍的神经网络中,相邻层的所有神经元之间都有连接,这称为\textbf{全连接}(fully-connected)。另外,我们用Affine层实现了全连接层。如果使用
这个Affine层,一个5层的全连接的神经网络就可以通过 \autoref{An example of a network based on a fully connected layer (Affine layer)}所示的网络结
构来实现。

\figures{An example of a network based on a fully connected layer (Affine layer)}

\figures{Examples of CNN-based networks}

如 \autoref{Examples of CNN-based networks} 所示CNN 中 新 增 了 Convolution 层 和 Pooling 层。CNN 的
层的连接顺序是“Convolution - ReLU -(Pooling)”
(Pooling 层有时会被省
略)。这可以理解为之前的“Affi ne - ReLU”连接被替换成了“Convolution -
ReLU -(Pooling)”连接。

还需要注意的是,靠近输出的层中使用了之前
的“Affi ne - ReLU”组合。此外,最后的输出层中使用了之前的“Affi ne -
Softmax”组合。这些都是一般的CNN中比较常见的结构。

\section{卷积层}
CNN中出现了一些特有的术语,比如填充、步幅等。此外,各层中传
递的数据是有形状的数据(比如,3维数据)。

\subsection{全连接层存在的问题}
之前介绍的全连接的神经网络中使用了全连接层(Affine层)。在全连接
层中,相邻层的神经元全部连接在一起,输出的数量可以任意决定。

\important{全连接层存在什么问题呢?那就是数据的形状被“忽视”了}。比如,输
入数据是图像时,图像通常是高、长、通道方向上的3维形状。但是,向全
连接层输入时,需要将3维数据拉平为1维数据。实际上,前面提到的使用
了 MNIST 数据集的例子中,输入图像就是 1 通道、高 28 像素、长 28 像素
的(1, 28, 28)形状,但却被排成1列,以784个数据的形式输入到最开始的
Affine层。

\begin{tcolorbox}
    图像是3维形状,这个形状中应该含有重要的空间信息。比如,空间上
    邻近的像素为相似的值、RGB的各个通道之间分别有密切的关联性、相距
    较远的像素之间没有什么关联等,3维形状中可能隐藏有值得提取的本质模
    式。但是,因为全连接层会忽视形状,将全部的输入数据作为相同的神经元
    (同一维度的神经元)处理,所以无法利用与形状相关的信息。
\end{tcolorbox}
而卷积层可以保持形状不变。因此,
在 CNN 中,可以(有可能)正确理解图像等具有形状的数据。

CNN 中,有时将卷积层的输入输出数据称为\textbf{特征图}(feature
map)。其中,卷积层的输入数据称为\textbf{输入特征图}(input feature map),输出
数据称为\textbf{输出特征图}(output feature map)。

\subsection{卷积运算}
卷积层进行的处理就是卷积运算。卷积运算相当于图像处理中的“滤波
器运算”(\autoref{Example of convolution operation})。
\figures{Example of convolution operation}

\figures{Calculation order of convolution operation}
对于输入数据,卷积运算以一定间隔滑动滤波器的窗口并应用。这里所
说的窗口是指 \autoref{Calculation order of convolution operation} 中灰色的$3 \times 3$的部分。如 \autoref{Calculation order of convolution operation} 所示,将各个位置上滤
波器的元素和输入的对应元素相乘,然后再求和(有时将这个计算称为\textbf{乘积累加运算})。然后,将这个结果保存到输出的对应位置。将这个过程在所有
位置都进行一遍,就可以得到卷积运算的输出。

在全连接的神经网络中,除了权重参数,还存在偏置。CNN中,滤波
器的参数就对应之前的权重。并且,CNN中也存在偏置。\autoref{Example of convolution operation} 的卷积运算
的例子一直展示到了应用滤波器的阶段。包含偏置的卷积运算的处理流如图 \autoref{The bias of the convolution operation} 所示。

\figures{The bias of the convolution operation}

如 \autoref{The bias of the convolution operation} 所示,向应用了滤波器的数据加上了偏置。偏置通常只有1个
($1 \times 1$)
(本例中,相对于应用了滤波器的4个数据,偏置只有1个),这个值
会被加到应用了滤波器的所有元素上。

\subsection{填充}
在进行卷积层的处理之前,有时要向输入数据的周围填入固定的数据(比
如0等),这称为\textbf{填充}(padding),是卷积运算中经常会用到的处理。

\figures{Filling processing of convolution operation}
如 \autoref{Filling processing of convolution operation} 所示,通过填充,大小为$(4, 4)$的输入数据变成了$(6, 6)$的形状。
然后,应用大小为$(3, 3)$的滤波器,生成了大小为$(4, 4)$的输出数据。这个例
子中将填充设成了1,不过填充的值也可以设置成2、3等任意的整数。

\begin{tcolorbox}
    \important{使用填充主要是为了调整输出的大小}。比如,对大小为$(4, 4)$的输入
    数据应用$(3, 3)$的滤波器时,输出大小变为$(2, 2)$,相当于输出大小
    比输入大小缩小了2个元素。这在反复进行多次卷积运算的深度网
    络中会成为问题。为什么呢?因为如果每次进行卷积运算都会缩小
    空间,那么在某个时刻输出大小就有可能变为1,导致无法再应用
    卷积运算。为了避免出现这样的情况,就要使用填充。在刚才的例
    子中,将填充的幅度设为1,那么相对于输入大小$(4, 4)$,输出大小
    也保持为原来的$(4, 4)$。因此,卷积运算就可以在保持空间大小不变
    的情况下将数据传给下一层。
\end{tcolorbox}

\subsection{步幅}
应用滤波器的位置间隔称为步幅(stride)。在 \autoref{Example of convolution operation with stride 2} 的例子中,对输入大小为$(7, 7)$的数据,以步幅2应用了滤波器。
通过将步幅设为2,输出大小变为$(3, 3)$。像这样,步幅可以指定应用滤波器
的间隔。

\figures{Example of convolution operation with stride 2}

综上,增大步幅后,输出大小会变小。而增大填充后,输出大小会变大。

假设输入大小为$(H, W)$,滤波器大小为$(FH, FW)$,输出大小为
$(OH, OW)$,填充为$P$,步幅为$S$。此时,输出大小可通过 \autoref{eq7.1} 进行计算。

\begin{equation}
    \label{eq7.1}
    \begin{aligned}
        OH & =\frac{H+2P-FH}{S}+1 \\
        OW & =\frac{W+2P-FW}{S}+1 \\
    \end{aligned}
\end{equation}