\chapter{误差反向传播法}
通过数值微分计算了神经网
络的权重参数的梯度(严格来说,是损失函数关于权重参数的梯度)。数值微
分虽然简单,也容易实现,但缺点是计算上比较费时间。本章我们将学习一
个能够高效计算权重参数的梯度的方法——误差反向传播法。
\section{计算图}
计算图将计算过程用图形表示出来。这里说的图形是数据结构图,通
过多个节点和边表示(连接节点的直线称为“边”)。

计算图通过节点和箭头表示计算过程。节点用$\bigcirc$表示,$\bigcirc$中是计算的内
容。将计算的中间结果写在箭头的上方,表示各个节点的计算结果从左向右
传递。

\figures{Based on the calculation graph to solve the answer}

虽然\autoref{Based on the calculation graph to solve the answer}中把“$\times 2$”“$\times 1.1$”等作为一个运算整体用○括起来了,不过
只用$\bigcirc$表示乘法运算“$\times$”也是可行的。此时,如\autoref{Based on the calculation graph to solve the answer2}所示,可以将“2”和
“1.1”分别作为变量“苹果的个数”和“消费税”标在$\bigcirc$外面。

\figures{Based on the calculation graph to solve the answer2}

用计算图解题的情况下,需要按如下流程进行。
\begin{enumerate}
    \item 构建计算图。
    \item 在计算图上,从左向右进行计算。
\end{enumerate}

这里的第2歩“从左向右进行计算”是一种正方向上的传播,简称为\textbf{正
    向传播}\marginpar[正向传播]{正向传播}(forward propagation)。正向传播是从计算图出发点到结束点的传播。
既然有正向传播这个名称,当然也可以考虑反向(从图上看的话,就是从右向左)
的传播。实际上,这种传播称为\textbf{反向传播}\marginpar[反向传播]{反向传播}(backward propagation)。反向传
播将在接下来的导数计算中发挥重要作用。

\subsection{局部计算}
计算图的特征是可以通过传递“局部计算”获得最终结果。“局部”这个
词的意思是“与自己相关的某个小范围”。局部计算是指,无论全局发生了什么,
都能只根据与自己相关的信息输出接下来的结果。
